\begin{frame}{Antes de Começar}
Estrutura de arquivos nesse projeto:
\begin{block}{Estrutura de Arquivos}
    \begin{itemize}
        \item beamerthemesrc \hspace{2pt} \% A pasta de temas.
        \item images \hspace{2pt} \% Coloque suas imagens aqui.
        \item header.tex \hspace{2pt} \% Coloque seus pacotes e comandos aqui.
        \item main.tex \hspace{2pt} \% Compile esse arquivo main.tex
        \item contents \hspace{2pt} \% Conteúdos dentro de main.tex
    \end{itemize}
\end{block}
\end{frame}

\begin{frame}[fragile]{Começando}
\framesubtitle{Selecionando o tema SINTEF}
Para começar trabalhando com \texttt{sintefbeamer}, comece um documento \LaTeX\ com o preâmbulo:
\begin{block}{SINTEF mínimo de um Documento Beamer}
\verb|\documentclass{beamer}|\\
\verb|\usepackage[brazil]{babel}
\usepackage[utf8]{inputenc}
\usepackage{lmodern}
\usepackage[T1]{fontenc}
\usetheme{src/sintef}
\usefonttheme[onlymath]{serif}
\titlebackground*{beamerthemesrc/assets/background}
%-------------add your packages here-------------
\usepackage{amsfonts,amsmath,oldgerm}



%-------------add your commands here-------------
\newcommand{\hrefcol}[2]{\textcolor{cyan}{\href{#1}{#2}}}
\newcommand{\testcolor}[1]{\colorbox{#1}{\textcolor{#1}{test}}~\texttt{#1}}


|\\
\verb|\begin{document}|\\
\verb|\begin{frame}{Hello, world!}|\\
\verb|\end{frame}|\\
\verb|\end{document}|\\
\end{block}
\end{frame}

\begin{frame}[fragile]{Título da Página}
Para definir uma página de título típica, você chama alguns comandos no preâmbulo:
\begin{block}{Comandos para a página de título}
\begin{verbatim}
\title{Título de Exemplo}
\subtitle{Subtítulo de Exemplo}
\author{Primeiro Autor, Segundo Autor}
\date{\today} % Também pode ser usado para nome da conferência &c.
\end{verbatim}
\end{block}
Você pode então escrever a página de título com \verb|\maketitle|.

Para definir uma \textbf{imagem de fundo} use o comando \verb|\titlebackground| antes de \verb|\maketitle|; seu único argumento é o nome (ou caminho) de um arquivo gráfico.

Se você usar a \textbf{versão com estrela} \verb|\titlebackground*|,  a imagem será cortada em uma exibição dividida no lado direito do slide do título.
\end{frame}

\begin{frame}[fragile]{Escrevendo um slide simples}
\framesubtitle{ É muito fácil!}
\begin{itemize}[<+->]
\item Um slide típico tem listas com marcadores
\item Estes podem ser descobertos em sequência
\end{itemize}
\begin{block}{Código para uma página com uma lista detalhada}<+->
\begin{verbatim}
\begin{frame}{Writing a Simple Slide}
  \framesubtitle{It's really easy!}
  \begin{itemize}[<+->]
    \item A typical slide has bulleted lists
    \item These can be uncovered in sequence
  \end{itemize}\end{frame}
\end{verbatim}
\end{block}
\end{frame}
